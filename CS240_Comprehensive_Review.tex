\documentclass{article}
\usepackage[utf8]{inputenc}
\usepackage{algpseudocode}
\usepackage{bookman}

\topmargin=-0.45in
\evensidemargin=0in
\oddsidemargin=0in
\textwidth=6.5in
\textheight=9.0in
\headsep=0.25in

\title{CS240 Comprehensive Review}
\author{Theo Park}
\date{3 May 2022}

\begin{document}

\maketitle

\section{Compiling and Linking}

\subsection{Gcc Flags}
\begin{itemize}
    \item \textit{-c} Compile file into object file
    \item \textit{-g} Debugging symbols
    \item \textbf{\textit{-Wall}} Include ALL Warning
    \item \textbf{\textit{-Werror}} Turn wanings into errors
    \item \textbf{\textit{-O1, -O2, -O3}} Optimize output code
    \item \textbf{\textit{-o filename}} Output to filename
    \item \textit{-ANSI} Adhere to ANSI std
    \item \textit{-std=C99} Adhere to C99 std
\end{itemize}

\subsection{Linking}
Object file contains binary code, symbol tables, and is a compiled form of a C module.
To make it a complete executable, one must link object files, with one of them containing main().

\section{File I/O}

\subsection{Essentials}
\begin{itemize}
    \item FILE *fopen(char *file\_name, char *mode);\\
    Modes are "r", "w", and "a" (append). Returns file ptr on success, NULL on unsuccess, so one must check the return val of fopen().
    \item int fclose(FILE *file\_pointer);\\
    It does not set the file ptr to NULL, so you have to manually set it to NULL. Return val check isn't necessary in this class.
    \item \textbf{int fprinf(FILE *stream, const char *format, \dots);}
    \item \textbf{int fscanf(FILE *stream, const char *format, \dots);}
    \item int access(char *file\_name, int mode);\\
    Used to check if file can be accessed in "R\_OK", "W\_OK", or "F\_OK" (check for existence) mode.
    \item int feof(FILE *file\_pointer);\\
    Returns non-zero if EOF reached.
    \item int ferror(FILE *file\_pointer);\\
    Returns 0 if error occurs (e.g disk space full).
\end{itemize}

\subsection{Notes with fscanf()}
\begin{itemize}
    \item Utilize \%[] (\%[0-9A-z] \%[\^A-z])
    \item \textbf{Field width specifier (e.g \%49s \%49[A-z]). Always one less than the buffer size to account for NUL terminator.}
    \item Assigns variables to pointers; use \& symbol for non-strings.
    \item Returns number of successfully read variables; Check for error using the return value.
\end{itemize}

\subsection{Random Access File I/O}
\begin{itemize}
    \item \textbf{int ftell(FILE *file\_pointer);}
    Returns current offset from the beginning of the file (SEEK\_SET) or -1 in case of error.
    \item \textbf{int fseek(FILE *fp, long int offset, int whence);}\\
    Whence values include
    \begin{itemize}
        \item SEEK\_SET: Offset relative to beginning of the file 
        \item SEEK\_CUR: Offset relative to the current position
        \item SEEK\_END: Offset relative to the end of the file
    \end{itemize}
    \item Example of finding how long the file is:
    \begin{algorithmic}
        \item fseek(fp, 0, SEEK\_END);
        \item int len = ftell(fp);
        \item fseek(fp, 0, SEEK\_SET);
    \end{algorithmic}
\end{itemize}

\section{Struct and Typedef}

\subsection{Some Syntax}
\begin{itemize}
    \item Typedef and struct definition:
    \begin{algorithmic}
        \item typedef struct my\_data \{\\
            $\indent$ int age;\\
        \} my\_data\_t;
    \end{algorithmic}
    \item Struct definition and declaration:
    \begin{algorithmic}
        \item struct my\_data \{\\
            $\indent$ int age;\\
        \} my\_var = \{ 19 \};
    \end{algorithmic}
\end{itemize}

\subsection{Declaration vs Definition}
Declaration is announcing the properties of var (no memory allocation), definition is allocating storages for a var.
\begin{itemize}
    \item Declaration:
    \begin{algorithmic}
        \item struct my\_data \{\\
            $\indent$ int age;\\
        \};
    \end{algorithmic}
    \item Definition and initialization:
    \begin{algorithmic}
        \item struct my\_data my\_var = \{ 19 \};
    \end{algorithmic}
\end{itemize}
Put pure declaration (struct, func prototype, extern) outside of the func, put definition inside func.

\subsection{Arrays in Struct}

\end{document}